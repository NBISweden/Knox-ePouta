% Language, diacritics and hyphenation
% Use English languages. 
\usepackage[english]{babel} 

% Font settings
\usepackage[utf8]{inputenc}
\usepackage[T1]{fontenc}
% \usepackage{times}
	
\usepackage{pifont}% http://ctan.org/pkg/pifont
%\usepackage{dingbat}
%\usepackage{boisik}% for a better hash symbol #

% Document links and bookmarks
\usepackage{url}
%\usepackage[pdftex,bookmarks]{hyperref}

%% -----------------------------------------------------------
%% For Tikz
%% -----------------------------------------------------------
\usepackage[version=latest]{pgf}
%\pgfrealjobname{main}

\usepackage{tikz}
\usetikzlibrary{%
  intersections,%
  decorations.pathmorphing,%
  shapes.symbols,%
  shapes.multipart,%
  positioning,%
  shadows,%
  backgrounds}

%% ================================
%% TikZ styles
%% ================================

\pgfdeclarelayer{my background} 
\pgfsetlayers{background,my background,main}

\tikzset{background rectangle/.style={rounded corners=1ex,draw=gray!5,thick,fill=gray!10,double}} % inner sep=1pt,bottom color=red!20,top color=white

%------------------------------------------

\tikzset{vm/.style={%
drop shadow,
rounded corners=0.5ex,thin,inner xsep=2pt,inner ysep=1ex,
rectangle split,rectangle split parts=2,
every two node part/.style={font=\tiny},
text width=12ex,align=center,
top color=blue!30,bottom color=blue!30,middle color=blue!10
}}

\tikzset{link/.style={%
fill=white,
draw=black,double,
inner xsep=4ex,rounded corners=1ex
}}
\tikzset{connector/.style={draw,circle,fill=white,inner sep=0pt,minimum size=3pt}}

\tikzset{nodename/.style={
draw,double,fill=white,rounded corners=0.5ex,
inner xsep=1ex,inner ysep=0.5ex,
anchor=center
}}

\makeatletter
\pgfdeclareshape{out-interface-shape}{
  \inheritsavedanchors[from=rectangle] % this is nearly a rectangle
  \inheritanchorborder[from=rectangle]
  \inheritanchor[from=rectangle]{center}
  \inheritanchor[from=rectangle]{north}
  \inheritanchor[from=rectangle]{south}
  \inheritanchor[from=rectangle]{west}
  \inheritanchor[from=rectangle]{east}
  \inheritanchor[from=rectangle]{south west}
  \inheritanchor[from=rectangle]{south east}
  \inheritanchor[from=rectangle]{north east}
  \inheritanchor[from=rectangle]{north west}
  \backgroundpath{% this is new
    % store lower right in xa/ya and upper right in xb/yb
    \southwest \pgf@xa=\pgf@x \pgf@ya=\pgf@y
    \northeast \pgf@xb=\pgf@x \pgf@yb=\pgf@y
    % construct main path
    \pgfsetcornersarced{\pgfpoint{0.5ex}{0.5ex}}
    \pgfpathmoveto{\pgfpoint{\pgf@xa}{\pgf@ya}}
    \pgfpathlineto{\pgfpoint{\pgf@xa}{\pgf@yb}}
    \pgfpathlineto{\pgfpoint{\pgf@xb}{\pgf@yb}}
    \pgfsetcornersarced{\pgfpoint{0pt}{0pt}}
    \pgfpathlineto{\pgfpoint{\pgf@xb}{\pgf@ya}}
    \pgfpathclose
 }
}
\makeatother

\tikzset{vlan-interface/.style={ rounded corners=0.5ex,%draw,
fill=blue!30,inner xsep=1ex,inner ysep=1ex,anchor=center,shape=rectangle,
}}

\tikzset{out-interface/.style={rounded corners=0.5ex,
fill=blue!30,inner xsep=1ex,inner ysep=1ex,anchor=center,shape=out-interface-shape,
}}



%------------------------------------------
%% Non atomic
\tikzset{looplabel/.style={scale=0.7,inner xsep=2pt,draw=gray!10,fill=white,rounded corners=2pt,anchor=center}}
 

%------------------------------------------

\tikzset{enumbullet/.style={draw,circle,double,inner sep=1pt}}
\tikzset{challenge/.style={draw,circle,double,scale=0.75,inner sep=1pt}}
\tikzset{subchallenge/.style={circle,fill=white,scale=0.6,inner sep=0.5pt,draw=black,very thin}}

\tikzset{myedge/.style={draw,shorten >=1pt,>=stealth',semithick}}
\tikzset{process/.style={circle,minimum width=2ex,inner sep=1pt,draw=blue!50,fill=blue!20,thick}}
\tikzset{mylabel/.style={inner xsep=2pt,draw=gray!10,fill=white,double,rounded corners=2pt,anchor=center}}
\tikzset{separation/.style={white,semithick}}

\tikzset{context/.style={rectangle,minimum width=3.2ex,minimum height=3.2ex,inner sep=0pt,fill=none,draw=black,thin,scale=0.6,anchor=center,fill=yellow!20!white}}
\tikzset{project/.style={draw,shorten >=1pt,>=stealth',very thin,blue!70!red}}
\tikzset{context-matrix/.style={matrix of nodes,inner xsep=0pt,column sep=1.5pt,column 1/.style={anchor=base}}}
%\tikzset{na-looplabel/.style={inner sep=2pt,draw=gray,fill=white,rounded corners=2pt,anchor=base,yshift=-2pt}}


%% -----------------------------------------------------------
%% Defs
%% -----------------------------------------------------------

%% Framing the lists
\usepackage[framemethod=TikZ]{mdframed}
\mdfdefinestyle{DazFrame}{%
    linecolor=gray!10!white,outerlinewidth=1pt,roundcorner=1ex,
    innertopmargin=1ex,innerrightmargin=2ex,innerbottommargin=1ex,innerleftmargin=1ex,
    backgroundcolor=white}

\newenvironment{dazenumerate}{%
  \renewcommand{\theenumi}{\protect\tikz[baseline=(n.base)]{\protect\node[enumbullet](n){\arabic{enumi}};}}
  \renewcommand{\labelenumi}{\theenumi}
  \begin{mdframed}[style=DazFrame]%
  \begin{enumerate}}{\end{enumerate}\end{mdframed}}

%% ---------------------------------------------
%\usepackage{verbatim}
\usepackage{fancyvrb}

%\newcommand*{\highlight}[1]{{\protect#1}}
\newcommand*{\highlight}[1]{\textcolor{red}{#1}}
\newcommand{\openbrace}{\{}
\newcommand{\closebrace}{\}}

\newcommand{\codeblock}[1]{%
%\begin{mdframed}[style=DazFrame]%
\begingroup\VerbatimInput[numbers=none,fontsize=\small,commentchar=!,commandchars=@\{\}]{scripts/#1}\endgroup%
%\end{mdframed}%
}
\newcommand{\codeline}[1]{\texttt{#1}}

\newcommand{\kw}[1]{\texttt{#1}}

%% ---------------------------------------------
%% Settings for enumerations and item lists
\usepackage{enumitem}
\setlist{noitemsep}
\setlist[enumerate,1]{itemsep=1ex}
%\setlist[enumerate]{align=right,labelindent=\parindent, leftmargin=*,widest*=4}
\setlist[enumerate]{align=right,leftmargin=*,widest*=4}
%\setlist[itemize]{labelindent=0pt,align=right,leftmargin=*}

%% ---------------------------------------------
%% Tables' style
\usepackage{hhline}
\usepackage{multirow}
\usepackage{multicol}
\usepackage{trimspaces} 
\makeatletter
\newcommand{\mytrim}[1]{\trim@post@space{#1}}
\makeatother

\newcommand{\resulttitle}[3]{%
\multicolumn{1}{c}{} & \multicolumn{2}{c}{\texttt{#1}}&\multicolumn{2}{c}{\texttt{#2}}& \multicolumn{2}{c}{\texttt{#3}}%
}
\newcommand{\resulttest}[6]{\mytrim{#1}s&{\scriptsize \mytrim{#2} MB/s}&\mytrim{#3}s&{\scriptsize \mytrim{#4} MB/s}&\mytrim{#5}s&{\scriptsize \mytrim{#6} MB/s}}

\newcommand{\resultpartition}[7]{%
\noindent\begin{center}
\begin{tabular}{r|rl||rl||rl|}
\resulttitle#1                                                    \\\hhline{~--||--||--}
\multirow{2}{*}{\ref{experiments:SOB:test:big}}   & \resulttest#2 \\
                                                  & \resulttest#3 \\\hhline{~==::==::==}
                \ref{experiments:SOB:test:dir}    & \resulttest#4 \\\hhline{~==::==::==}
              \ref{experiments:SOB:test:random}   & \resulttest#5 \\\hhline{~==::==::==}
\multirow{2}{*}{\ref{experiments:SOB:test:cache}} & \resulttest#6 \\
                                                  & \resulttest#7 \\\hhline{~--||--||--}
\multicolumn{1}{c}{} & \multicolumn{1}{r}{\tiny Time} & \multicolumn{1}{l}{\tiny Bandwidth} & \multicolumn{1}{r}{\tiny Time} & \multicolumn{1}{l}{\tiny Bandwidth} & \multicolumn{1}{r}{\tiny Time} & \multicolumn{1}{l}{\tiny Bandwidth} \\
\end{tabular}
\end{center}%
}


%% ---------------------------------------------
%% MACROS
%% ---------------------------------------------

\newcommand{\leftpointingfinger}{\ding{43}}
\newcommand{\ie}{i.e.\ }
\newcommand{\eg}{e.g.\ }
\newcommand{\ip}[1]{\texttt{#1}}
\newcommand{\param}[1]{\textless{}#1\textgreater{}}
\newcommand{\uuid}{\param{...}}

% \usepackage{amsfonts,amscd,amsmath,amsxtra}
\usepackage{amssymb}
%\usepackage{mathabx} 
%\usepackage{stmaryrd}
\newcommand{\vmconnect}{\ensuremath{\rightleftharpoons}}
%\newcommand{\veth}{\ensuremath{\nleftrightarrow}}
%\newcommand{\veth}{\ensuremath{\leftrightsquigarrow}}
\newcommand{\veth}{\ensuremath{\mathrel{\ooalign{$\leftrightsquigarrow$\cr\hidewidth{\scriptsize|}\hidewidth\cr}}}}


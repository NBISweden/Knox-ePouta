% Language, diacritics and hyphenation
% Use English languages. 
\usepackage[english]{babel} 

% Font settings
\usepackage[utf8]{inputenc}
\usepackage[T1]{fontenc}
% \usepackage{times}
	
% Enable scaling of images on import
%\usepackage[pdftex]{graphicx}
\usepackage[pdftex]{graphics}

% Tables
\usepackage{booktabs}
\usepackage{tabularx}

%% -----------------------------------------------------------
%% Other packages
%% -----------------------------------------------------------
\usepackage{hhline}
\usepackage{multirow}
\usepackage{multicol}
%\usepackage{longtable}

\usepackage{trimspaces} 

\usepackage{wrapfig}
\usepackage{enumitem}
%\usepackage{paralist}

\usepackage{nth} % For 1st, 2nd, 3rd 4th, ...

\usepackage{import} % to include files using relative directories
\usepackage{xcolor}
\usepackage{amsfonts,amscd,amssymb,amsmath,amsxtra}
\usepackage{mathtools}
\usepackage{etoolbox}
\usepackage{ifthen}
\usepackage{stmaryrd}
\usepackage{url}
%\usepackage{skull}
\usepackage{fancyvrb}

\usepackage{esint} % For the squared contour integral

\usepackage{pifont}% http://ctan.org/pkg/pifont
%\usepackage{dingbat}

%\usepackage{fancyhdr,layout,appendix,subfigure}

% Document links and bookmarks
\usepackage[pdftex,bookmarks]{hyperref}


%% -----------------------------------------------------------
%% For Graphics
%% -----------------------------------------------------------
%\usepackage[hiresbb]{graphicx}
% \usepackage{graphics}
% \DeclareGraphicsExtensions{.png,.xbb,.pdf,.jpg}
% \DeclareGraphicsRule{.png}{eps}{.xbb}{}

%% -----------------------------------------------------------
%% For Tikz
%% -----------------------------------------------------------
\usepackage[version=latest]{pgf}
\pgfrealjobname{main}

\usepackage{tikz}
\usetikzlibrary{%
  arrows,%
  calc,%
  %decorations,%
  decorations.pathmorphing,%
  %decorations.pathreplacing,%
  %calendar,%
  chains,%
  fit,%
  %shapes,%
  shapes.geometric,%
  shapes.misc,%
  %shapes.symbols,%
  %shapes.arrows,%
  %shapes.callouts,%
  shapes.multipart,%
  %backgrounds,%
  matrix,%
  %fadings,%
  %through,%
  positioning,%
  %scopes,%
  %decorations.shapes,%
  %decorations.pathmorphing,%
  %decorations.text,%
  shadows,%
  trees,%
  %snakes,% use decorations instead
  petri,%
  automata,%
  backgrounds,%
  patterns%
}


%% -----------------------------------------------------------
%% Algorithms
%% -----------------------------------------------------------

\usepackage{verbatim}


% \usepackage[nofillcomment,noend,linesnumbered,noline,ruled]{algorithm2e}
% %\usepackage[noend]{algorithmic}
% \usepackage{listings}

% \lstdefinestyle{custom}{
%   showspaces=false,               % show spaces adding particular underscores
%   showstringspaces=false,         % underline spaces within strings
%   showtabs=false,                 % show tabs within strings adding particular underscores
%   %belowcaptionskip=1\baselineskip,
%   breaklines=true,
%   frame=BT,                       % Lines above and below
%   %frame=B,                        % Lines below only
%   %xleftmargin=\parindent,
%   language=bash,
%   basicstyle=\footnotesize\ttfamily,
%   keywordstyle=\bfseries,%\color{green!40!black},
%   commentstyle=\itshape\color{purple},
%   %identifierstyle=\color{blue},
%   %stringstyle=\color{orange},
%   escapechar=@,
%   %escapeinside={\%*}{*)}          % if you want to add a comment within your code 
%   mathescape=true,
%   captionpos=b,                   % sets the caption-position to bottom
%   breaklines=true,                % sets automatic line breaking
%   breakatwhitespace=false,        % sets if automatic breaks should only happen at whitespace
% }

% % \lstdefinestyle{numbers}{numbers=left, stepnumber=1, numberstyle=\tiny, numbersep=2pt}
% % \lstdefinestyle{nonumbers}{numbers=none}

% % \RecustomVerbatimEnvironment{Verbatim}{Verbatim}{
% %   fontfamily=helvetica,
% %   numbers=left,
% %   numbersep=2pt,
% %   stepnumber=1,
% %   firstnumber=0,
% %   numberblanklines=false,
% %   commandchars=\\\[\],
% %   codes={\catcode`$=3}
% % }
% \fvset{fontfamily=helvetica,numbers=left,numbersep=5pt,stepnumber=1,firstnumber=0,numberblanklines=true,commandchars=\\\[\],codes={\catcode`$=3\catcode`^=7}}
% %codes={\catcode`$=3}
% \renewcommand{\theFancyVerbLine}{\tiny \arabic{FancyVerbLine}}

%% -----------------------------------------------------------
%% Definitions and Tikz styles
%% -----------------------------------------------------------
%% ================================
%% TikZ styles
%% ================================

\pgfdeclarelayer{my background} 
\pgfsetlayers{background,my background,main}

\tikzset{background rectangle/.style={rounded corners=1ex,draw=gray!5,thick,fill=gray!10,double}} % inner sep=1pt,bottom color=red!20,top color=white

%------------------------------------------

\tikzset{vm/.style={%
preaction={draw},% does only draw the contour, and not the separation
%drop shadow,
rounded corners=0.5ex,thin,inner xsep=2pt,inner ysep=1ex,
rectangle split,rectangle split parts=2,
every two node part/.style={font=\tiny},
text width=12ex,align=center,
top color=blue!30,bottom color=blue!30,middle color=blue!10
}}

\tikzset{link/.style={%
fill=white,
draw=black,double,
inner xsep=4ex,rounded corners=1ex
}}
\tikzset{connector/.style={draw,circle,fill=white,inner sep=0pt,minimum size=3pt}}

\tikzset{nodename/.style={
draw,double,fill=white,rounded corners=0.5ex,
inner xsep=1ex,inner ysep=0.5ex,
anchor=center
}}

\makeatletter
\pgfdeclareshape{out-interface-shape}{
  \inheritsavedanchors[from=rectangle] % this is nearly a rectangle
  \inheritanchorborder[from=rectangle]
  \inheritanchor[from=rectangle]{center}
  \inheritanchor[from=rectangle]{north}
  \inheritanchor[from=rectangle]{south}
  \inheritanchor[from=rectangle]{west}
  \inheritanchor[from=rectangle]{east}
  \inheritanchor[from=rectangle]{south west}
  \inheritanchor[from=rectangle]{south east}
  \inheritanchor[from=rectangle]{north east}
  \inheritanchor[from=rectangle]{north west}
  \backgroundpath{% this is new
    % store lower right in xa/ya and upper right in xb/yb
    \southwest \pgf@xa=\pgf@x \pgf@ya=\pgf@y
    \northeast \pgf@xb=\pgf@x \pgf@yb=\pgf@y
    % construct main path
    \pgfsetcornersarced{\pgfpoint{0.5ex}{0.5ex}}
    \pgfpathmoveto{\pgfpoint{\pgf@xa}{\pgf@ya}}
    \pgfpathlineto{\pgfpoint{\pgf@xa}{\pgf@yb}}
    \pgfpathlineto{\pgfpoint{\pgf@xb}{\pgf@yb}}
    \pgfsetcornersarced{\pgfpoint{0pt}{0pt}}
    \pgfpathlineto{\pgfpoint{\pgf@xb}{\pgf@ya}}
    \pgfpathclose
 }
}
\makeatother

\tikzset{vlan-interface/.style={ rounded corners=0.5ex,%draw,
fill=blue!30,inner xsep=1ex,inner ysep=1ex,anchor=center,shape=rectangle,
}}

\tikzset{out-interface/.style={rounded corners=0.5ex,
fill=blue!30,inner xsep=1ex,inner ysep=1ex,anchor=center,shape=out-interface-shape,
}}

\tikzset{bridge/.style={
preaction={draw,line width=5pt,black},
draw,line width=2pt,white,
}}

\tikzset{iptables/.style={
draw=green!50!black, fill=green!30,
cloud,cloud puffs=15,cloud puff arc=120,cloud ignores aspect,
scale=0.5,
}}

%% -----------------------------------------------------------
\newcommand\builddir{\detokenize{/Users/daz/Workspace/NBIS/Knox-ePouta/repo/paper/_build}}

%\ifunderprogress

\mdfdefinestyle{CommentFrame}{%
    linecolor=green!10,outerlinewidth=1pt,roundcorner=1ex,
    innertopmargin=1em,innerrightmargin=2ex,innerbottommargin=1em,innerleftmargin=2ex,
    backgroundcolor=green!5!white}
\mdfdefinestyle{TodoFrame}{%
    linecolor=red!10,outerlinewidth=1pt,roundcorner=1ex,
    innertopmargin=1em,innerrightmargin=2ex,innerbottommargin=1em,innerleftmargin=2ex,
    backgroundcolor=red!5!white}

\newenvironment{comment}{\vspace{1em}\begin{mdframed}[style=CommentFrame]}{\end{mdframed}\vspace{1em}}
\newenvironment{todo}{\vspace{1em}\begin{mdframed}[style=TodoFrame]\noindent\hrulefill\raisebox{-0.5ex}{\ TODO\ }\hrulefill\par\vspace{1em}}{\end{mdframed}\vspace{1em}}

\else
%\newsavebox\myframeb %% Saving and throwing away the content
%\newenvironment{myframe}[1]{\setbox\myframeb\hbox\bgroup}{\egroup}

% \usepackage{comment}   % for the comment environment
% \newenvironment{todo}{\begin{comment}}{\end{comment}}

\usepackage{environ}
\NewEnviron{killcontents}{}
\newenvironment{todo}{\killcontents}{\endkillcontents}
\newenvironment{comment}{\killcontents}{\endkillcontents}
\fi




%% ---------------------------------------------
%% Framing the lists
\usepackage[framemethod=TikZ]{mdframed}
\mdfdefinestyle{DazFrame}{%
    linecolor=gray!10!white,outerlinewidth=1pt,roundcorner=1ex,
    innertopmargin=1ex,innerrightmargin=2ex,innerbottommargin=1ex,innerleftmargin=1ex,
    backgroundcolor=white}

\newenvironment{tests}{%
  \renewcommand{\theenumi}{\protect\tikz[baseline=(n.base)]{\protect\node[enumbullet](n){\arabic{enumi}};}}
  \renewcommand{\labelenumi}{\theenumi}
  \begin{mdframed}[style=DazFrame]%
  \begin{enumerate}}{\end{enumerate}\end{mdframed}}

%% ---------------------------------------------
\newcommand{\codeblock}[1]{%
%\begin{mdframed}[style=DazFrame]%
\begingroup\VerbatimInput[numbers=none]{scripts/#1}\endgroup%
%\end{mdframed}%
}
\newcommand{\codeline}[1]{\texttt{#1}}


%% Settings for enumerations and item lists
\setlist{noitemsep}
\setlist[enumerate,1]{itemsep=1ex}
%\setlist[enumerate]{align=right,labelindent=\parindent, leftmargin=*,widest*=4}
\setlist[enumerate]{align=right,leftmargin=*,widest*=4}
%\setlist[itemize]{labelindent=0pt,align=right,leftmargin=*}

\makeatletter
\newcommand{\mytrim}[1]{\trim@post@space{#1}}
\makeatother

\newcommand{\resulttitle}[3]{%
\multicolumn{1}{c}{} & \multicolumn{2}{c}{\texttt{#1}}&\multicolumn{2}{c}{\texttt{#2}}& \multicolumn{2}{c}{\texttt{#3}}%
}
\newcommand{\resulttest}[6]{\mytrim{#1}s&{\scriptsize \mytrim{#2} MB/s}&\mytrim{#3}s&{\scriptsize \mytrim{#4} MB/s}&\mytrim{#5}s&{\scriptsize \mytrim{#6} MB/s}}

\newcommand{\resultpartition}[7]{%
\noindent\begin{center}
\begin{tabular}{r|rl||rl||rl|}
\resulttitle#1                                                    \\\hhline{~--||--||--}
\multirow{2}{*}{\ref{experiments:SOB:test:big}}   & \resulttest#2 \\
                                                  & \resulttest#3 \\\hhline{~==::==::==}
                \ref{experiments:SOB:test:dir}    & \resulttest#4 \\\hhline{~==::==::==}
              \ref{experiments:SOB:test:random}   & \resulttest#5 \\\hhline{~==::==::==}
\multirow{2}{*}{\ref{experiments:SOB:test:cache}} & \resulttest#6 \\
                                                  & \resulttest#7 \\\hhline{~--||--||--}
\multicolumn{1}{c}{} & \multicolumn{1}{r}{\tiny Time} & \multicolumn{1}{l}{\tiny Bandwidth} & \multicolumn{1}{r}{\tiny Time} & \multicolumn{1}{l}{\tiny Bandwidth} & \multicolumn{1}{r}{\tiny Time} & \multicolumn{1}{l}{\tiny Bandwidth} \\
\end{tabular}
\end{center}%
}


%% ---------------------------------------------
%% MACROS
%% ---------------------------------------------

\newcommand{\leftpointingfinger}{\ding{43}}
\newcommand{\vmconnect}{\ensuremath{\rightleftharpoons}}


\newcommand{\ip}[1]{\texttt{#1}}

\newcommand{\ie}{i.e.\ }
\newcommand{\eg}{e.g.\ }

 %% Moving it after since defs can use styles

%% -----------------------------------------------------------
%% So far so good
\newcommand*{\sofarsogood}{\par\bigskip\noindent\hrulefill\begingroup\tiny\raisebox{-0.5ex}{\ SO FAR SO GOOD\ }\endgroup\hrulefill\par\bigskip}
\newcommand*{\sfsg}{\sofarsogood\endinput}
\newcommand*{\cutafter}{\endinput}
%\newcommand*{\cutpage}{\noindent\makebox[\linewidth]{\rule{\paperwidth}{2pt}}}

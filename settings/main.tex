% Language, diacritics and hyphenation
% Use English languages. 
\usepackage[english]{babel} 

% Font settings
\usepackage[utf8]{inputenc}
\usepackage[T1]{fontenc}
% \usepackage{times}
	
% Enable scaling of images on import
%\usepackage[pdftex]{graphicx}
\usepackage[pdftex]{graphics}

% Tables
\usepackage{booktabs}
\usepackage{tabularx}

%% -----------------------------------------------------------
%% Other packages
%% -----------------------------------------------------------
\usepackage{hhline}
\usepackage{multirow}
\usepackage{multicol}
%\usepackage{longtable}

\usepackage{trimspaces} 

\usepackage{wrapfig}
\usepackage{enumitem}
%\usepackage{paralist}

\usepackage{nth} % For 1st, 2nd, 3rd 4th, ...

\usepackage{import} % to include files using relative directories
\usepackage{xcolor}
\usepackage{amsfonts,amscd,amssymb,amsmath,amsxtra}
\usepackage{mathtools}
\usepackage{etoolbox}
\usepackage{ifthen}
\usepackage{stmaryrd}
\usepackage{url}
%\usepackage{skull}
\usepackage{fancyvrb}

\usepackage{esint} % For the squared contour integral

\usepackage{pifont}% http://ctan.org/pkg/pifont
%\usepackage{dingbat}

%\usepackage{fancyhdr,layout,appendix,subfigure}

% Document links and bookmarks
\usepackage[pdftex,bookmarks]{hyperref}


%% -----------------------------------------------------------
%% For Graphics
%% -----------------------------------------------------------
%\usepackage[hiresbb]{graphicx}
% \usepackage{graphics}
% \DeclareGraphicsExtensions{.png,.xbb,.pdf,.jpg}
% \DeclareGraphicsRule{.png}{eps}{.xbb}{}

%% -----------------------------------------------------------
%% For Tikz
%% -----------------------------------------------------------
\usepackage[version=latest]{pgf}
\pgfrealjobname{main}

\usepackage{tikz}
\usetikzlibrary{%
  arrows,%
  calc,%
  %decorations,%
  decorations.pathmorphing,%
  %decorations.pathreplacing,%
  %calendar,%
  chains,%
  fit,%
  %shapes,%
  shapes.geometric,%
  shapes.misc,%
  %shapes.symbols,%
  %shapes.arrows,%
  %shapes.callouts,%
  shapes.multipart,%
  %backgrounds,%
  matrix,%
  %fadings,%
  %through,%
  positioning,%
  %scopes,%
  %decorations.shapes,%
  %decorations.pathmorphing,%
  %decorations.text,%
  shadows,%
  trees,%
  %snakes,% use decorations instead
  petri,%
  automata,%
  backgrounds,%
  patterns%
}


%% -----------------------------------------------------------
%% Algorithms
%% -----------------------------------------------------------

\usepackage{verbatim}


% \usepackage[nofillcomment,noend,linesnumbered,noline,ruled]{algorithm2e}
% %\usepackage[noend]{algorithmic}
% \usepackage{listings}

% \lstdefinestyle{custom}{
%   showspaces=false,               % show spaces adding particular underscores
%   showstringspaces=false,         % underline spaces within strings
%   showtabs=false,                 % show tabs within strings adding particular underscores
%   %belowcaptionskip=1\baselineskip,
%   breaklines=true,
%   frame=BT,                       % Lines above and below
%   %frame=B,                        % Lines below only
%   %xleftmargin=\parindent,
%   language=bash,
%   basicstyle=\footnotesize\ttfamily,
%   keywordstyle=\bfseries,%\color{green!40!black},
%   commentstyle=\itshape\color{purple},
%   %identifierstyle=\color{blue},
%   %stringstyle=\color{orange},
%   escapechar=@,
%   %escapeinside={\%*}{*)}          % if you want to add a comment within your code 
%   mathescape=true,
%   captionpos=b,                   % sets the caption-position to bottom
%   breaklines=true,                % sets automatic line breaking
%   breakatwhitespace=false,        % sets if automatic breaks should only happen at whitespace
% }

% % \lstdefinestyle{numbers}{numbers=left, stepnumber=1, numberstyle=\tiny, numbersep=2pt}
% % \lstdefinestyle{nonumbers}{numbers=none}

% % \RecustomVerbatimEnvironment{Verbatim}{Verbatim}{
% %   fontfamily=helvetica,
% %   numbers=left,
% %   numbersep=2pt,
% %   stepnumber=1,
% %   firstnumber=0,
% %   numberblanklines=false,
% %   commandchars=\\\[\],
% %   codes={\catcode`$=3}
% % }
% \fvset{fontfamily=helvetica,numbers=left,numbersep=5pt,stepnumber=1,firstnumber=0,numberblanklines=true,commandchars=\\\[\],codes={\catcode`$=3\catcode`^=7}}
% %codes={\catcode`$=3}
% \renewcommand{\theFancyVerbLine}{\tiny \arabic{FancyVerbLine}}

%% -----------------------------------------------------------
%% Definitions and Tikz styles
%% -----------------------------------------------------------
%% ================================
%% TikZ styles
%% ================================

\pgfdeclarelayer{my background} 
\pgfsetlayers{background,my background,main}

\tikzset{background rectangle/.style={rounded corners=1ex,draw=gray!5,thick,fill=gray!10,double}} % inner sep=1pt,bottom color=red!20,top color=white

%------------------------------------------

\tikzset{vm/.style={%
drop shadow,
rounded corners=0.5ex,thin,inner xsep=2pt,inner ysep=1ex,
rectangle split,rectangle split parts=2,
every two node part/.style={font=\tiny},
text width=12ex,align=center,
top color=blue!30,bottom color=blue!30,middle color=blue!10
}}

\tikzset{link/.style={%
fill=white,
draw=black,double,
inner xsep=4ex,rounded corners=1ex
}}
\tikzset{connector/.style={draw,circle,fill=white,inner sep=0pt,minimum size=3pt}}

\tikzset{nodename/.style={
draw,double,fill=white,rounded corners=0.5ex,
inner xsep=1ex,inner ysep=0.5ex,
anchor=center
}}

\makeatletter
\pgfdeclareshape{out-interface-shape}{
  \inheritsavedanchors[from=rectangle] % this is nearly a rectangle
  \inheritanchorborder[from=rectangle]
  \inheritanchor[from=rectangle]{center}
  \inheritanchor[from=rectangle]{north}
  \inheritanchor[from=rectangle]{south}
  \inheritanchor[from=rectangle]{west}
  \inheritanchor[from=rectangle]{east}
  \inheritanchor[from=rectangle]{south west}
  \inheritanchor[from=rectangle]{south east}
  \inheritanchor[from=rectangle]{north east}
  \inheritanchor[from=rectangle]{north west}
  \backgroundpath{% this is new
    % store lower right in xa/ya and upper right in xb/yb
    \southwest \pgf@xa=\pgf@x \pgf@ya=\pgf@y
    \northeast \pgf@xb=\pgf@x \pgf@yb=\pgf@y
    % construct main path
    \pgfsetcornersarced{\pgfpoint{0.5ex}{0.5ex}}
    \pgfpathmoveto{\pgfpoint{\pgf@xa}{\pgf@ya}}
    \pgfpathlineto{\pgfpoint{\pgf@xa}{\pgf@yb}}
    \pgfpathlineto{\pgfpoint{\pgf@xb}{\pgf@yb}}
    \pgfsetcornersarced{\pgfpoint{0pt}{0pt}}
    \pgfpathlineto{\pgfpoint{\pgf@xb}{\pgf@ya}}
    \pgfpathclose
 }
}
\makeatother

\tikzset{vlan-interface/.style={ rounded corners=0.5ex,%draw,
fill=blue!30,inner xsep=1ex,inner ysep=1ex,anchor=center,shape=rectangle,
}}

\tikzset{out-interface/.style={rounded corners=0.5ex,
fill=blue!30,inner xsep=1ex,inner ysep=1ex,anchor=center,shape=out-interface-shape,
}}



%------------------------------------------
%% Non atomic
\tikzset{looplabel/.style={scale=0.7,inner xsep=2pt,draw=gray!10,fill=white,rounded corners=2pt,anchor=center}}
 

%------------------------------------------

\tikzset{enumbullet/.style={draw,circle,double,inner sep=1pt}}
\tikzset{challenge/.style={draw,circle,double,scale=0.75,inner sep=1pt}}
\tikzset{subchallenge/.style={circle,fill=white,scale=0.6,inner sep=0.5pt,draw=black,very thin}}

\tikzset{myedge/.style={draw,shorten >=1pt,>=stealth',semithick}}
\tikzset{process/.style={circle,minimum width=2ex,inner sep=1pt,draw=blue!50,fill=blue!20,thick}}
\tikzset{mylabel/.style={inner xsep=2pt,draw=gray!10,fill=white,double,rounded corners=2pt,anchor=center}}
\tikzset{separation/.style={white,semithick}}

\tikzset{context/.style={rectangle,minimum width=3.2ex,minimum height=3.2ex,inner sep=0pt,fill=none,draw=black,thin,scale=0.6,anchor=center,fill=yellow!20!white}}
\tikzset{project/.style={draw,shorten >=1pt,>=stealth',very thin,blue!70!red}}
\tikzset{context-matrix/.style={matrix of nodes,inner xsep=0pt,column sep=1.5pt,column 1/.style={anchor=base}}}
%\tikzset{na-looplabel/.style={inner sep=2pt,draw=gray,fill=white,rounded corners=2pt,anchor=base,yshift=-2pt}}

%% ---------------------------------------------
%% Framing the lists
\usepackage[framemethod=TikZ]{mdframed}
\mdfdefinestyle{DazFrame}{%
    linecolor=gray!10!white,outerlinewidth=1pt,roundcorner=1ex,
    innertopmargin=1ex,innerrightmargin=2ex,innerbottommargin=1ex,innerleftmargin=1ex,
    backgroundcolor=white}

%% -----------------------------------------------------------
\newcommand\builddir{\detokenize{/Users/kk/Work/Development/Knox-ePouta/_build}}

\ifunderprogress

\mdfdefinestyle{CommentFrame}{%
    linecolor=green!10,outerlinewidth=1pt,roundcorner=1ex,
    innertopmargin=1em,innerrightmargin=2ex,innerbottommargin=1em,innerleftmargin=2ex,
    backgroundcolor=green!5!white}
\mdfdefinestyle{TodoFrame}{%
    linecolor=red!10,outerlinewidth=1pt,roundcorner=1ex,
    innertopmargin=1em,innerrightmargin=2ex,innerbottommargin=1em,innerleftmargin=2ex,
    backgroundcolor=red!5!white}

\newenvironment{comment}{\vspace{1em}\begin{mdframed}[style=CommentFrame]}{\end{mdframed}\vspace{1em}}
\newenvironment{todo}{\vspace{1em}\begin{mdframed}[style=TodoFrame]\noindent\hrulefill\raisebox{-0.5ex}{\ TODO\ }\hrulefill\par\vspace{1em}}{\end{mdframed}\vspace{1em}}

\else
%\newsavebox\myframeb %% Saving and throwing away the content
%\newenvironment{myframe}[1]{\setbox\myframeb\hbox\bgroup}{\egroup}

% \usepackage{comment}   % for the comment environment
% \newenvironment{todo}{\begin{comment}}{\end{comment}}

\usepackage{environ}
\NewEnviron{killcontents}{}
\newenvironment{todo}{\killcontents}{\endkillcontents}
\newenvironment{comment}{\killcontents}{\endkillcontents}
\fi


%% -----------------------------------------------------------
%% So far so good
\newcommand*{\sofarsogood}{\ifunderprogress\par\bigskip\noindent\hrulefill\begingroup\tiny\raisebox{-0.5ex}{\ SO FAR SO GOOD\ }\endgroup\hrulefill\par\bigskip\fi}
\newcommand*{\sfsg}{\sofarsogood\endinput}
\newcommand*{\cutafter}{\endinput}
%\newcommand*{\cutpage}{\noindent\makebox[\linewidth]{\rule{\paperwidth}{2pt}}}


%% ---------------------------------------------
\newenvironment{statement}{\begin{quote}\raggedleft}{\end{quote}}

%% Settings for enumerations and item lists
\setlist{noitemsep}
\setlist[enumerate,1]{itemsep=1ex}
%\setlist[enumerate]{align=right,labelindent=\parindent, leftmargin=*,widest*=4}
\setlist[enumerate]{align=right,leftmargin=*,widest*=4}
%\setlist[itemize]{labelindent=0pt,align=right,leftmargin=*}

\newcommand{\point}[1]{\ifnotikz\arabic{enumi}.\else\protect\tikz[baseline=(n.base)]{\protect\node[enumbullet](n){#1};}\fi}

\newenvironment{strategy}{%
  \renewcommand{\labelenumi}{\point{\arabic{enumi}}}
  % No need to redefine \theenumi since there is no cross-referencing
  \begin{mdframed}[style=DazFrame]\begin{enumerate}}{\end{enumerate}\end{mdframed}}

\newenvironment{challenges}{%
  \renewcommand{\labelenumi}{\theenumi}%
  \renewcommand{\theenumi}{\ifnotikz\Alph{enumi}\else\protect\tikz[baseline=(n.base)]{\protect\path node[challenge](n){\Alph{enumi}};}\fi}%
  % \smallskip\hrule\smallskip%
  \begin{mdframed}[style=DazFrame]%
    % \begin{enumerate}[leftmargin=0pt]
    \begin{enumerate}%
    }{\end{enumerate}%
    % \smallskip\hrule\smallskip
  \end{mdframed}%
}
\newenvironment{subchallenges}{%
  \renewcommand{\labelenumii}{\theenumii}%
  \renewcommand{\theenumi}{}%
  \renewcommand{\theenumii}{\ifnotikz\Alph{enumi}.\arabic{enumii}\else\protect\tikz[baseline=(n.base)]{\protect\path node[challenge](n){\Alph{enumi}} node[subchallenge,right=-1pt of n.south east]{\arabic{enumii}};}\fi}%
  \begin{enumerate}}{\end{enumerate}}

%% ---------------------------------------------
%% Wrapfigures
%% ---------------------------------------------
%\setlength{\intextsep}{1em}
\setlength{\intextsep}{0pt}
%\setlength{\columnsep}{1ex}

%% ---------------------------------------------
%% ---------------------------------------------
%% Math stuff
%% ---------------------------------------------

\newcommand{\set}[1]{\{#1\}}
\newcommand*{\setcomp}[2]{\{{#1}\mathrel{}\mid\mathrel{}#2\}}

\newcommand{\nat}{\ensuremath{\mathbb N}}
\newcommand{\reals}{\ensuremath{\mathbb R}}
\newcommand{\sizeof}[1]{|#1|}
\newcommand{\union}{\cup}
\newcommand{\Union}{\bigcup}
%\newcommand{\minsetunion}{\sqcup}
\newcommand{\range}[2]{\llbracket{#1}{,}{#2}\rrbracket} %% {,} otherwise I get some spacing after the ','

% \newcommand{\updateby}[2]{\ensuremath{\left[#1\leftarrow#2\right]}}

%% ---------------------------------------------
%% Parameterized systems
%% ---------------------------------------------

\newcommand{\parsys}{\ensuremath{\mathcal P}}
\newcommand{\locs}{\ensuremath{Q}}
\newcommand{\rules}{\ensuremath{\Delta}}

\newcommand{\witnesses}{S}
\newcommand{\quantrule}[5]{ \mathbf{if}\ {#3}~j\,{#4}\,i:\,{#5}\ \mathbf{then}\ {#1}\trans{#2}}
\newcommand{\quantify}{\mathbb Q}

\newcommand{\confs}{\ensuremath{\mathcal C}}
\newcommand{\trans}{\ensuremath{\rightarrow}}
\newcommand{\transof}[1]{\stackrel{#1}{\trans}}
\newcommand{\transys}{\ensuremath{\mathcal T}}

\newcommand{\src}{\mathtt{src}}
\newcommand{\dst}{\mathtt{dst}}

\newcommand{\Bad}{\ensuremath{\mathcal{B}}}
\newcommand{\minbad}{\ensuremath{\Bad_{min}}}
\newcommand{\Reach}{\ensuremath{\mathcal{R}}}
\newcommand{\Inits}{\ensuremath{\mathcal{I}}}

%\newcommand{\domain}{\ensuremath{\mathcal{D}}}
\newcommand{\entails}{\preccurlyeq}
%\newcommand{\preorder}{\leqslant}%\vartriangleleft

\newcommand{\upcl}[1]{\ensuremath{\mathopen{\uparrow}{#1}}} %% Upward-Closure
\newcommand{\downcl}[1]{\ensuremath{\mathopen{\downarrow}{#1}}} %% Downward-Closure
% \newcommand{\upcl}[1]{\ensuremath{{#1}^\cup}} %% Upward-Closure
% \newcommand{\downcl}[1]{\ensuremath{{#1}^\cap}} %% Downward-Closure
% \newcommand{\upcl}[1]{\ensuremath{\lfloor{#1}\rfloor}} %% Upward-Closure
% \newcommand{\downcl}[1]{\ensuremath{\lceil{#1}\rceil}} %% Downward-Closure

%\newcommand{\gen}[1]{min(#1)}
\newcommand{\gen}[1]{gen(#1)}
%\newcommand{\compl}[1]{\neg{#1}}
\newcommand{\compl}[1]{\overline{#1}}

\newcommand{\forrule}[5]{\ensuremath{\mathbf{if~foreach}\ j\mathrel{#1}i: {#2} \ \mathbf{then}\ {#3}\trans{#4}\ \mathbf{else}\ {#3}\trans{#5}}}

%\newcommand{\placeholder}{\bot}
\newcommand{\placeholder}{\bullet}

\newcommand{\rel}{\ensuremath{\mathcal R}}
\newcommand{\distance}[1]{\ensuremath{\|#1\|}}

%% ---------------------------------------------
%% View abstraction
%% ---------------------------------------------
\newcommand{\dcl}[1]{\ensuremath{\lceil{#1}\rceil}} %% Downward-Closure

\newcommand{\subword}{\sqsubseteq}

\newcommand{\Abs}{\alpha} 
\newcommand{\Conc}{\gamma}

\newcommand{\Absof}[1]{\Abs_{#1}}
\newcommand{\Concof}[1]{\Conc_{#1}}
\newcommand{\Concoflim}[2]{\oint_{#1}^{#2}}

\newcommand {\views}{\mathcal{V}}
\newcommand {\viewsof}[1]{\views_{#1}}

\newcommand {\proj}[2]{\Pi_{#1}(#2)}

\newcommand {\post}{\mathit{post}}
\newcommand {\spost}{\mathit{spost}}
\newcommand {\apost}[1]{{\mathit{Apost}}_#1}
\newcommand {\sdelta}{\delta^\#}

\newcommand{\entailedby}{\succcurlyeq}
\newcommand{\minsetof}[1]{\lfloor #1 \rfloor}
\newcommand{\minunion}{\sqcup}

\newcommand{\base}{\mathtt{base}}
\newcommand{\ctx}{\mathtt{ctx}}

% \usepackage{wasysym}
% \newcommand{\aConcoflim}[2]{\ensuremath{\mathbin{\ooalign{\hspace{.2ex}\raisebox{.15ex}{\scalebox{.7}{\wasylozenge}}\cr$\int$\cr}}_{#1}^{#2}}}
%\newcommand{\aConcoflim}[2]{\ensuremath{\mathbin{\ooalign{\hspace{.2ex}\raisebox{.15ex}{\scalebox{.6}{$\square$}}\cr$\int_{#1}^{#2}$\cr}}}}
\newcommand{\aConcoflim}[2]{\sqint_{#1}^{#2}}
\newcommand{\isbad}{\ensuremath{\mathtt{bad}}}

\newcommand{\tick}{\checkmark}
\newcommand{\unticked}{\rho}%\xi \chi
\renewcommand{\next}{\mathit{next}}

\newcommand{\makehighgroup}[4][]{
  \draw[blue!70!red,#1] (#2.north west) ++(0,1mm) -- +(0,1mm) -- ([yshift=2mm]#3.north east) coordinate[midway](n#4) -- +(0,-1mm);
}
\newcommand{\makelowgroup}[4][]{
  \draw[#1] (#2.south west) ++(0,-1mm) -- +(0,-0.5mm) -- ([yshift=-1.5mm]#3.south east) coordinate[midway](p#4) -- +(0,0.5mm);
}
%% ---------------------------------------------
%% Proofs
%% ---------------------------------------------
\newcommand{\strue}{\ensuremath{\mathtt{true}}}
\newcommand{\sfalse}{\ensuremath{\mathtt{false}}}
\newcommand{\Unsafe}{\ensuremath{\mathtt{Unsafe}}}
\newcommand{\Safe}{\ensuremath{\mathtt{Safe}}}

\newcommand{\pos}{\ensuremath{\mathbb{P}}}
%\newcommand{\viewof}[2]{\ensuremath{\mathit{view_{#2}(#1)}}}
\newcommand{\viewof}[2]{\ensuremath{\mathit{view(#1,#2)}}}

%% ---------------------------------------------
%% Trees
%% ---------------------------------------------
\newcommand{\subtree}{\preceq}
\newcommand{\node}{v}
\newcommand{\nodes}{N}
\newcommand{\tree}{t}
\newcommand{\embed}{e}
\newcommand{\img}{\iota}

%% ---------------------------------------------
%% Rings
%% ---------------------------------------------
\newcommand{\csubword}{\trianglelefteq}

%% ---------------------------------------------
%% Experiments
%% ---------------------------------------------
\newcommand{\adc}{\ensuremath{\forall_\exists}}
\newcommand{\cmark}{\ding{51}}%
\newcommand{\xmark}{\ding{55}}%





 %% Moving it after since defs can use styles

%% -----------------------------------------------------------
%% So far so good
\newcommand*{\sofarsogood}{\par\bigskip\noindent\hrulefill\begingroup\tiny\raisebox{-0.5ex}{\ SO FAR SO GOOD\ }\endgroup\hrulefill\par\bigskip}
\newcommand*{\sfsg}{\sofarsogood\endinput}
\newcommand*{\cutafter}{\endinput}
%\newcommand*{\cutpage}{\noindent\makebox[\linewidth]{\rule{\paperwidth}{2pt}}}

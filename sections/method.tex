%% ====================================================================
\section{Method}
\label{section:method}
%% ====================================================================

The Mosler secure computing environment~\cite{mosler} enables
scientists to carry out research on \emph{sensitive data}, \eg
personal data related to health.
%
For each project on the system, a set of virtual machines (VMs) is
booted, which the users access via a secured connection. All
communication is encrypted.

When a VM is started, it ``books'' predefined CPU resources and disk
space and when all the hardware cores are booked, we say that the
cloud cluster is full. In such a case, we are interested in extending
the set of VMs with external ones running in another cluster, and in
our case, the extra cluster is located in another country.

\begin{figure}[b]
  \centering
  \tikz\draw circle(2cm) node{wow...impressive!};
  %% \beginpgfgraphicnamed{overview}
\begin{tikzpicture}
  [vm/.append style={drop shadow}]  
  %\draw[help lines] (0,0) grid (3,2);

  \draw[rounded corners,thick]%
  (0,0) to[bend right=15]
  coordinate[pos=0]    (vm1-epouta)
  coordinate[pos=0.25] (vm2-epouta)
  coordinate[pos=0.5]  (vm3-epouta)
  coordinate[pos=0.75] (dhcp-epouta)
  ++(-1.5,-5) to
  node[pos=0.3,link] (link) {1 Gb/s}
  coordinate[pos=0.7] (router)
  coordinate[pos=0.87] (dhcp-knox)
  ++(6,0) to[bend right=15]
  coordinate[pos=0.01] (supernode)
  coordinate[pos=0.25] (storage)
  coordinate[pos=1]    (vm1-knox)
  coordinate[pos=0.75] (vm2-knox)
  coordinate[pos=0.5]  (vm3-knox)
  ++(0.5,5);

  \path[every pin/.style={vm},pin distance=2ex,every pin edge/.style={thick,bend right=10,auto}]
  % 
  (vm1-epouta)  node[connector,pin=left:{VM$_1$\nodepart{two}10.101.0.21}] {}
  (vm2-epouta)  node[connector,pin=left:{VM$_2$\nodepart{two}10.101.0.22}] {}
  (vm3-epouta)  node[connector,pin=left:{VM$_3$\nodepart{two}10.101.0.23}] {}
  (dhcp-epouta) node[connector,pin=left:{DHCP\nodepart{two}10.101.0.3}] {}
  (router)      node[connector,pin=above:{Router\nodepart{two}10.101.0.1}] {}
  (dhcp-knox)   node[connector,pin=below:{DHCP\nodepart{two}10.101.128.0}] {}
  (supernode)   node[connector,pin=right:{Supernode\nodepart{two}10.101.128.100}] {}
  (storage)     node[connector,pin=right:{Storage\nodepart{two}10.101.128.104}] {}
  (vm1-knox)    node[connector,pin=right:{VM$_1$\nodepart{two}10.101.128.101}] {}
  (vm2-knox)    node[connector,pin=right:{VM$_2$\nodepart{two}10.101.128.102}] {}
  (vm3-knox)    node[connector,pin=right:{VM$_3$\nodepart{two}10.101.128.103}] {}
  ;


  \path[name path=border] (link) -- ++(60:65mm) node[left=4ex,pos=0.5](fin){Finland} node[right=6ex,pos=0.4](swe){Sweden};
  \path [name path=limit] (vm1-epouta) -- (vm1-knox);
  \path [name intersections={of=border and limit, by=x},] (link) -- (x);
  
  \path (link.center) ++(left:3cm) coordinate(link');
  \path (link.center) ++(right:7cm) coordinate(link'');
  \path (x) ++(left:6cm) coordinate(x');
  \path (x) ++(right:4cm) coordinate(x'');
  
  \begin{pgfonlayer}{my background}
    \clip (-3,0) rectangle +(9,-5);

    \shade[right color=blue!20,left color=white,decorate,decoration={snake,segment length=6ex, amplitude=0.5ex}]%
    (link.center) -- (x) -- (x') -- (link') -- cycle;

    \shade[left color=yellow!50,right color=white,decorate,decoration={snake,segment length=6ex, amplitude=0.5ex}]%
    (link.center) -- (x) -- (x'') -- (link'') -- cycle;
  \end{pgfonlayer}



  %% SWE Flag
  \begin{scope}[shift={(swe.west)},scale=0.1]
    %% Poteau
    \draw[fill=black] (-.2,0) to [bend right] (.2,0) -- (.2,8) to [bend left] (-.2,8) -- cycle;
    \draw[fill=black] (0,8) circle (.4) ;
    \begin{scope}
      %% Flag contour
      \clip
      (0.2,7.6)
      .. controls +(60:20mm) and +(-160:60mm) ..
      coordinate[pos=.55] (t1) coordinate[pos=.65] (t2)
      (8,9)
      to[out=-85 ,in=85]
      coordinate[pos=.3] (r1) coordinate[pos=.45] (r2)
      (7.5,4) 
      .. controls +(-140:30mm) and +(20:30mm) .. 
      coordinate[pos=.7] (b1) coordinate[pos=.6] (b2)
      (.2,3)
      to [out =85,in=-80]
      coordinate[pos=.7] (l1) coordinate[pos=.5] (l2)
      cycle;
      %% Background
      \path[fill=blue] 
      (0.2,7.6)
      .. controls +(60:20mm) and +(-160:60mm) ..
      (8,9)
      to[out=-85 ,in=85]
      (7.5,4) 
      .. controls +(-140:30mm) and +(20:30mm) .. 
      (.2,3)
      to [out =85,in=-80]
      cycle;
      %% Stripes
      \path[fill=yellow] (l1) .. controls +(60:20mm) and +(-160:60mm) .. (r1) -- (r2) .. controls +(-160:60mm) and +(60:20mm) .. (l2) -- cycle;
      \path[fill=yellow] (t1) to[in=85,out=-80] (b1) -- (b2) to[out=85,in=-80] (t2) -- cycle;
    \end{scope}
  \end{scope}


  %% FIN Flag
  \begin{scope}[shift={(fin.west)},scale=0.1]
    %% Poteau
    \draw[fill=black] (-.2,0) to [bend right] (.2,0) -- (.2,8) to [bend left] (-.2,8) -- cycle;
    \draw[fill=black] (0,8) circle (.4) ;
    \begin{scope}
      %% Flag contour
      \clip
      (0.2,7.6)
      .. controls +(60:20mm) and +(-160:60mm) ..
      coordinate[pos=.55] (t1) coordinate[pos=.65] (t2)
      (8,9)
      to[out=-85 ,in=85]
      coordinate[pos=.3] (r1) coordinate[pos=.45] (r2)
      (7.5,4) 
      .. controls +(-140:30mm) and +(20:30mm) .. 
      coordinate[pos=.7] (b1) coordinate[pos=.6] (b2)
      (.2,3)
      to [out =85,in=-80]
      coordinate[pos=.7] (l1) coordinate[pos=.5] (l2)
      cycle;
      %% Background
      \path[fill=white] 
      (0.2,7.6)
      .. controls +(60:20mm) and +(-160:60mm) ..
      (8,9)
      to[out=-85 ,in=85]
      (7.5,4) 
      .. controls +(-140:30mm) and +(20:30mm) .. 
      (.2,3)
      to [out =85,in=-80]
      cycle;
      %% Stripes
      \path[fill=blue] (l1) .. controls +(60:20mm) and +(-160:60mm) .. (r1) -- (r2) .. controls +(-160:60mm) and +(60:20mm) .. (l2) -- cycle;
      \path[fill=blue] (t1) to[in=85,out=-80] (b1) -- (b2) to[out=85,in=-80] (t2) -- cycle;
    \end{scope}
  \end{scope}




\end{tikzpicture}
% \endpgfgraphicnamed

  \caption{Overview of the project resources}
  \label{figure:overview}
\end{figure}

We build a temporary cluster in Sweden, which we refer to as Knox, and
use the ePouta cluster~\cite{epouta} in Finland to extend a project
running on Knox. As a first step, we isolate the project from any
other project, but do not encrypt the communications, \ie the
environment manipulates non-sensitive data.
%
The VMs infrastructure is shaped as in
Figure~\ref{figure:overview}. We boot on Knox:
\begin{itemize}
\item a Virtual Router,
\item a DHCP server, providing network settings to the Knox VMs,
\item 3 VMs for computations
\item a storage node
\item a \emph{supernode} where we connect and run the tests
\end{itemize}
On ePouta, we boot:
\begin{itemize}
\item a DHCP server, providing network settings to the ePouta VMs,
\item 3 VMs for computations
\end{itemize}

We connect the two clusters with a 1GB/s fiber-link. We provide a
local network between the VMs, which can communicate across borders
transparently.
%
Note that the link might be shared by other projects, but VMs from
other projects cannot reach the above-mentioned VMs.
%
This isolation is provided by VLAN encapsulation.

%%%%%%%%%%%%%%%%%%%%%%%%%%%%%%%%%

We aim at running realistic workflows on the extended cluster, and see
whether we can notice a significant slowdown when computations are
scheduled on VMs across borders.
%
We choose to run the \href{https://github.com/SciLifeLab/CAW}{Cancer
  Analysis Workflow}~\cite{caw} from SciLifeLab and the
\href{https://github.com/NBISweden/wgs-structvar}{Whole Genome
  Sequencing Structural Variation Pipeline}~\cite{caw} from
\href{http://www.nbis.se}{NBIS}.
%
We denote them here CAW and WGS, respectively.
%
CAW is a pipeline for variant calling of tumor data and WGS is a
pipeline for predicting and annotating structural variations in the
human genome.
%
Yet, we only use non-sensitive data, here in the first step.

%%%%%%%%%%%%%%%%%%%%%%%%%%%%
Moreover, in this approach, where the computations might be scheduled
in Finland, while the data have to reside in Sweden, we suspect that
disk access and even the network itself would be a bottleneck. So, we
run some further tests to stress both aspects.
%
We use the program SOB~\cite{sob} to stress the filesystem, by reading
and writing an arbitrary number of files in different chunk sizes.
%
SOB also outputs useful performance numbers.
%
In order to test the network, we use iPerf3~\cite{iperf}, a tool for
active measurements of the maximum achievable bandwidth on IP
networks.
%
For each run, the tool reports the bandwidth, loss, and several other
parameters related to how well the link was used.
%
We instantiate a data communication between two VMs, one on Knox and
one on ePouta and test the connection.
%
We then repeat the same test with three connections in parallel over
pairwise-connected VMs.


%%%%%%%%%%%%%%%%%%%%%%%%%%%%
After running the tests, described in
Section~\ref{section:experiments}, we notice no significant slowdown
on the computations for the pipelines, and the stress-tests reveal
that (i)~disk~access is not slower and (ii)~the~link is almost fully
used.
%
Computations and tests do not notice whether they are performed in
Finland or in Sweden.
%
In other words, this is a very positive outcome where data traffic and
computations are carried out as if the extended VMs were all located
in the same country.
%
Computations are actually running slightly faster in Finland, since
the hardware in ePouta is better than the one in Knox.
%
It is probably possible to fine-tune kernel settings
to get even further performance and a seamless connection.
%
The remaining step dealing with sensitive data can thus be ignored.
%
You can refer to the
\href{https://github.com/NBISweden/Knox-ePouta}{NBIS GitHub
  repository}~\cite{nbis-knox-epouta} for the source code, in order to
reproduce the setup and re-run the tests.
%
It is worth noting that the link capacity must be adjusted in case the
number of VMs communicating through the link increases.

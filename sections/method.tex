%% ====================================================================
\section{Method}
\label{section:method}
%% ====================================================================

\cutafter

This section describes in great details how we connected a set of
virtual machines (VMs) running on Knox, a cloud cluster in Sweden, to
another set of VMs running on ePouta, a cloud cluster in Finland. The
two clusters are managed by
\href{http://docs.openstack.org/liberty/install-guide-ubuntu/}{Openstack
  Liberty}.

We start with the low-level component. There is a 1GB/s fiber-link
between Knox and ePouta with a dedicated network. This means that the
link might be shared, but only the VMs on that network can communicate
through it. This security is provided by VLAN encapsulation. In our
case, the VLAN tag is 1203.

We use the following range of IPs: \ip{10.101.0.0/16} for network
communication between the VMs. We split the address range in two
disjoint parts. For that, we use the third number of the
\ip{10.101.0.0/16} CIDR. The VMs in ePouta will have an IP where the
third number starts in binary notation with a \texttt{0}, while, for the
VMs in Knox, the third number will start with a \texttt{1}. The
exception is the virtual router.

The network settings use the following components:
\begin{itemize}
\item a Virtual Router on Knox (with IP: \ip{10.101.0.1/16})
\item a DHCP server on Knox (with IP: \ip{10.101.128.0/16})
\item a DHCP server on ePouta (with IP: \ip{10.101.0.3/16})
\item the Neutron linuxbridges plugin (on Knox) with VLAN capabilities
\item a set of VMs on ePouta (with IPs from \ip{10.101.0.4/16} to
  \ip{10.101.127.255/16})
\item a set of VMs on Knox (with IPs from \ip{10.101.128.1/16} to
  \ip{10.101.128.254/16})
\end{itemize}

The DHCP server on Knox (resp. ePouta) provides network information for
the VMs on Knox (resp. ePouta).

\cutafter


In order to test the connection between countries, 
we set up a dedicated network, \ie the link
is shared by other machines, but no other machine are connected to
that network.

Between Knox and ePouta, we set up a dedicated network, \ie the link
is shared by other machines, but no other machine are connected to
that network.

we ran two
realistic workflows and stress-tested the disc access and the network
link.
%
The outcome was surprising: we did not detect any significant slowdown
when using resources from either clusters, \ie computations do not
notice whether they are performed in Finland or in Sweden.
%
This is a very positive outcome as it is now possible to carry on with
workflows dealing with sensitive data.



We present in Section~\ref{section:method} a general overview of our
method




In order to test the connection between countries, we built a
temporary cloud cluster in Sweden, called Knox, and connected it to
the resources of ePouta, a secure cloud cluster in Finland, via a
fiber-link with a capacity of 1GB/s.

Between Knox and ePouta, we set up a dedicated network, \ie the link
is shared by other machines, but no other machine are connected to
that network.

We were interested in running realistic workflows. We choose to run
the \href{https://github.com/SciLifeLab/CAW}{Cancer Analysis Workflow}
(CAW) from SciLifeLab and the
\href{https://github.com/NBISweden/wgs-structvar}{Whole Genome
Sequencing Structural Variation Pipeline} (WGS) from
\href{http://www.nbis.se}{NBIS} as a first step. The results were
surprising: we did not detect any significant slowdown when using
resources from either clusters.

We suspected that disk access or even the network itself would be a
bottleneck in this setup. So, as a second step, we stress-tested both
aspects and noticed even more surprising results: the disk accesses
are not slower at all, and the link can be used almost to 100\%!

In other words, computations and tests do not notice whether they are
performed in Finland or in Sweden. Computations were actually even
faster running in Finland, since the hardware in ePouta is better than
the one in Knox. It is probably possible to fine-tune the settings to
get even further performance and a seamless connection. This is a very
positive outcome as we can now carry on with workflows dealing with
sensitive data. You can refer the
\href{https://github.com/NBISweden/Knox-ePouta}{NBIS GitHub
repository} for further information, or see
\href{https://NBISweden.github.io/Knox-ePouta/informal/}{an informal
presentation}.


% \beginpgfgraphicnamed{overview}
\begin{tikzpicture}
  [vm/.append style={drop shadow}]  
  %\draw[help lines] (0,0) grid (3,2);

  \draw[rounded corners,thick]%
  (0,0) to[bend right=15]
  coordinate[pos=0]    (vm1-epouta)
  coordinate[pos=0.25] (vm2-epouta)
  coordinate[pos=0.5]  (vm3-epouta)
  coordinate[pos=0.75] (dhcp-epouta)
  ++(-1.5,-5) to
  node[pos=0.3,link] (link) {1 Gb/s}
  coordinate[pos=0.7] (router)
  coordinate[pos=0.87] (dhcp-knox)
  ++(6,0) to[bend right=15]
  coordinate[pos=0.01] (supernode)
  coordinate[pos=0.25] (storage)
  coordinate[pos=1]    (vm1-knox)
  coordinate[pos=0.75] (vm2-knox)
  coordinate[pos=0.5]  (vm3-knox)
  ++(0.5,5);

  \path[every pin/.style={vm},pin distance=2ex,every pin edge/.style={thick,bend right=10,auto}]
  % 
  (vm1-epouta)  node[connector,pin=left:{VM$_1$\nodepart{two}10.101.0.21}] {}
  (vm2-epouta)  node[connector,pin=left:{VM$_2$\nodepart{two}10.101.0.22}] {}
  (vm3-epouta)  node[connector,pin=left:{VM$_3$\nodepart{two}10.101.0.23}] {}
  (dhcp-epouta) node[connector,pin=left:{DHCP\nodepart{two}10.101.0.3}] {}
  (router)      node[connector,pin=above:{Router\nodepart{two}10.101.0.1}] {}
  (dhcp-knox)   node[connector,pin=below:{DHCP\nodepart{two}10.101.128.0}] {}
  (supernode)   node[connector,pin=right:{Supernode\nodepart{two}10.101.128.100}] {}
  (storage)     node[connector,pin=right:{Storage\nodepart{two}10.101.128.104}] {}
  (vm1-knox)    node[connector,pin=right:{VM$_1$\nodepart{two}10.101.128.101}] {}
  (vm2-knox)    node[connector,pin=right:{VM$_2$\nodepart{two}10.101.128.102}] {}
  (vm3-knox)    node[connector,pin=right:{VM$_3$\nodepart{two}10.101.128.103}] {}
  ;


  \path[name path=border] (link) -- ++(60:65mm) node[left=4ex,pos=0.5](fin){Finland} node[right=6ex,pos=0.4](swe){Sweden};
  \path [name path=limit] (vm1-epouta) -- (vm1-knox);
  \path [name intersections={of=border and limit, by=x},] (link) -- (x);
  
  \path (link.center) ++(left:3cm) coordinate(link');
  \path (link.center) ++(right:7cm) coordinate(link'');
  \path (x) ++(left:6cm) coordinate(x');
  \path (x) ++(right:4cm) coordinate(x'');
  
  \begin{pgfonlayer}{my background}
    \clip (-3,0) rectangle +(9,-5);

    \shade[right color=blue!20,left color=white,decorate,decoration={snake,segment length=6ex, amplitude=0.5ex}]%
    (link.center) -- (x) -- (x') -- (link') -- cycle;

    \shade[left color=yellow!50,right color=white,decorate,decoration={snake,segment length=6ex, amplitude=0.5ex}]%
    (link.center) -- (x) -- (x'') -- (link'') -- cycle;
  \end{pgfonlayer}



  %% SWE Flag
  \begin{scope}[shift={(swe.west)},scale=0.1]
    %% Poteau
    \draw[fill=black] (-.2,0) to [bend right] (.2,0) -- (.2,8) to [bend left] (-.2,8) -- cycle;
    \draw[fill=black] (0,8) circle (.4) ;
    \begin{scope}
      %% Flag contour
      \clip
      (0.2,7.6)
      .. controls +(60:20mm) and +(-160:60mm) ..
      coordinate[pos=.55] (t1) coordinate[pos=.65] (t2)
      (8,9)
      to[out=-85 ,in=85]
      coordinate[pos=.3] (r1) coordinate[pos=.45] (r2)
      (7.5,4) 
      .. controls +(-140:30mm) and +(20:30mm) .. 
      coordinate[pos=.7] (b1) coordinate[pos=.6] (b2)
      (.2,3)
      to [out =85,in=-80]
      coordinate[pos=.7] (l1) coordinate[pos=.5] (l2)
      cycle;
      %% Background
      \path[fill=blue] 
      (0.2,7.6)
      .. controls +(60:20mm) and +(-160:60mm) ..
      (8,9)
      to[out=-85 ,in=85]
      (7.5,4) 
      .. controls +(-140:30mm) and +(20:30mm) .. 
      (.2,3)
      to [out =85,in=-80]
      cycle;
      %% Stripes
      \path[fill=yellow] (l1) .. controls +(60:20mm) and +(-160:60mm) .. (r1) -- (r2) .. controls +(-160:60mm) and +(60:20mm) .. (l2) -- cycle;
      \path[fill=yellow] (t1) to[in=85,out=-80] (b1) -- (b2) to[out=85,in=-80] (t2) -- cycle;
    \end{scope}
  \end{scope}


  %% FIN Flag
  \begin{scope}[shift={(fin.west)},scale=0.1]
    %% Poteau
    \draw[fill=black] (-.2,0) to [bend right] (.2,0) -- (.2,8) to [bend left] (-.2,8) -- cycle;
    \draw[fill=black] (0,8) circle (.4) ;
    \begin{scope}
      %% Flag contour
      \clip
      (0.2,7.6)
      .. controls +(60:20mm) and +(-160:60mm) ..
      coordinate[pos=.55] (t1) coordinate[pos=.65] (t2)
      (8,9)
      to[out=-85 ,in=85]
      coordinate[pos=.3] (r1) coordinate[pos=.45] (r2)
      (7.5,4) 
      .. controls +(-140:30mm) and +(20:30mm) .. 
      coordinate[pos=.7] (b1) coordinate[pos=.6] (b2)
      (.2,3)
      to [out =85,in=-80]
      coordinate[pos=.7] (l1) coordinate[pos=.5] (l2)
      cycle;
      %% Background
      \path[fill=white] 
      (0.2,7.6)
      .. controls +(60:20mm) and +(-160:60mm) ..
      (8,9)
      to[out=-85 ,in=85]
      (7.5,4) 
      .. controls +(-140:30mm) and +(20:30mm) .. 
      (.2,3)
      to [out =85,in=-80]
      cycle;
      %% Stripes
      \path[fill=blue] (l1) .. controls +(60:20mm) and +(-160:60mm) .. (r1) -- (r2) .. controls +(-160:60mm) and +(60:20mm) .. (l2) -- cycle;
      \path[fill=blue] (t1) to[in=85,out=-80] (b1) -- (b2) to[out=85,in=-80] (t2) -- cycle;
    \end{scope}
  \end{scope}




\end{tikzpicture}
% \endpgfgraphicnamed


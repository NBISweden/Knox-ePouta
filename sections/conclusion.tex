%%%%%%%%%%%%%%%%%%%%%%%%%%%%%%%%%%
\section{Conclusion}
\label{section:conclusion}
%%%%%%%%%%%%%%%%%%%%%%%%%%%%%%%%%%

Summary...
\vspace{1cm}

\paragraph{Future Work.}
%
It is probably possible to tweak the NFS settings, or the TCP settings
in the kernel, to gain even further speed.
%
It would be interesting to scale up the solution to
\textbf{many-many-many} nodes in ePouta and some nodes in Knox, to see
%how much the link can be shared.
at which point the communication link gets saturated.
%
Furthermore, it is worth updating the Slurm and NFS solution with
other tools, to improve disk access, such as (i)~the use object
storage or (ii)~the use a separate disk volume attached to the storage
VM, rather than an ephemeral disk on file (\ie the default in
\codeline{libvirt}).

\paragraph{Remarks.}
%
Broadcast traffic is still forwarded to all interfaces on VLAN 1203
and therefore to all VMs. An improvment would be to use OpenVSwitch to
learn about MAC addresses and skip physical nodes that don't host any
VMs on that project. That will improve East-West traffic. An
alternative is to distribute the router using DVR (not available when
using the Linux Bridge mechanism).


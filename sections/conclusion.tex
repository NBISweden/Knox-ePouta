%%%%%%%%%%%%%%%%%%%%%%%%%%%%%%%%%%
\section{Conclusion}
\label{section:conclusion}
%%%%%%%%%%%%%%%%%%%%%%%%%%%%%%%%%%

As part of the \href{https://wiki.neic.no/wiki/Tryggve}{Tryggve
  project}, we presented a method to extend the local computational
resources, when they are all busy, with cloud resources of partners in
other countries.
%
This is an interesting alternative to buying more hardware, because we
stress-tested the network and disc access and we noticed that the
performance degradation is close to neglectable.
%
Moreover, we ran realistic pipelines on non-sensitive --but real--
data successfully using compute resources that stretch across national
borders.
%
The current architecture is dependend on the link speed between the computer centers. And obvisously the link need to be scaled accordingly if the compute sharing is needed to be extended. But we do not think that it needs to be a linear scaling. Since its a temporary solution to just fix some computing tops not handled by the local centre. We think that it will be OK user wise to have some lag in network speed. The user will get his work done faster than waiting on local resources. 
%

\paragraph{Future Work.}
%
It is probably possible to tweak the NFS settings, or the TCP settings
in the kernel, to gain even further performance.
%
It would be interesting to scale up the solution to \emph{many} nodes
in ePouta and \emph{some} nodes in Knox, to see
%how much the link can be shared.
at which point the communication link gets saturated (and vice-versa).
%
Furthermore, it is worth updating the Slurm and NFS solutions with
other tools, to improve disk access, such as (i)~the use object
storage or (ii)~the use a separate disk volume attached to the storage
VM, rather than an ephemeral disk on file (\ie the default in
\codeline{libvirt}).

Regarding the network implementation, broadcast traffic is still
forwarded to all interfaces on VLAN 1203 and therefore to all VMs. An
improvment would be to use OpenVSwitch to learn about MAC addresses
and skip physical nodes that don't host any VMs on that project. That
will improve \emph{east-west} traffic. Another alternative is to
distribute the router using DVR, which is however not available when
using the Linux Bridge mechanism.


%% ====================================================================
\section{Introduction}
\label{section:intro}

When a cluster runs at full capacity, all the newly scheduled jobs
have to wait. In case this happens often, it is necessary to scale up
the infrastructure for more computations and more data transfers. To
this end, it is of course possible to buy more hardware, \ie more
compute nodes, more disks and more network switches. However, this
might be an expensive solution.

The \href{https://wiki.neic.no/wiki/Tryggve}{Tryggve
  project}~\cite{tryggve} focused therefore on an alternative
approach, where the cloud resources of other countries are solicited.
%
If they could be \emph{``borrowed for a while''}, we extend the local
computational power with such extra nodes residing in other countries.
%
An immediate technical challenge with such a solution is whether a
connection across borders is even feasible, or if there is a
penalizing latency. We can imagine the scenario where computations
happen in one country, while the data are located in another country,
and it would require high bandwidth for data transfers.

Initially, we were interested in running \emph{data-sensitive}
analysis using compute resources that stretch across borders.
%
As a first step, we instantiated a few compute nodes in one country,
Sweden, and a few other nodes in another country, Finland. We
connected them transparently to the same network such that they are
the only nodes communicating over that network, for security. We then
performed tests to stress network and disc accesses, and ran two
realistic workflows over non-sensitive data.

Since our method yielded no significant slowdown in that step, we
could simplify the problem and ignore the next steps related to
experiments over sensitive data.
%
Note that there is another issue related to the transfer of sensitive
data between countries. This legal matter is an ongoing work within
the Tryggve project, and left out of this paper.

We present a general overview of our method in
Section~\ref{section:method}, and the tests we performed in
Section~\ref{section:experiments}.
%
In Section~\ref{section:implementation}, we describe in greater
details the technical setup using Openstack clouds.

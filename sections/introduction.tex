%% ====================================================================
\section{Introduction}
\label{section:intro}

When a computational cluster runs at full capacity, any additional jobs will
have to wait. If this happens often, it may be necessary to scale up
the infrastructure to cope with the additional requirements on computations and rate of data transfers. To
this end, it is of course possible to buy more hardware, \ie more
compute nodes, more disks and more network switches. However, this
might be an expensive solution.

The \href{https://wiki.neic.no/wiki/Tryggve}{Tryggve
  project}~\cite{tryggve} focuses on an alternative
approach, where the cloud resources of collaborators in other countries are solicited.
%
If the resources could be \emph{``borrowed for a while''}, we extend the local
computational power with such extra nodes residing in other countries.
%
An immediate technical challenge with such a solution is whether a
connection across borders is even feasible, or if there is a
penalizing latency. We can imagine the scenario where the computations
happen in one country, while the data are located in another country,
and that would require high bandwidth for data transfers.

Initially, we are interested in running \emph{data-sensitive}
analysis using compute resources that stretch across national borders.
%
As a first step, we instantiate a few compute nodes in one country,
Sweden, and a few other nodes in another country, Finland. We
connect them transparently to the same virtual network so that they are
the only nodes communicating over that network, for security. We then
perform tests to stress network and disc accesses, and run two
realistic workflows on non-sensitive data.

Since our method yields no significant slowdown in that step, we
can simplify the problem and ignore the next steps related to
experiments utilizing sensitive data.
%
Note that there is another issue regarding the transfer of sensitive
data between countries. This legal matter is an ongoing work within
the Tryggve project, and left out of this paper.

We present a general overview of our method in
Section~\ref{section:method}, and the tests we performed in
Section~\ref{section:experiments}.
%
In Section~\ref{section:implementation}, we describe in greater
details the technical setup using Openstack clouds.
%
Finally, we conclude and give directions for future work in
Section~\ref{section:conclusion}.

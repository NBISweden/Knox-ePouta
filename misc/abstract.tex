\begin{abstract}
When a compute cluster runs at full capacity, all the newly scheduled
jobs have to wait. In case this happens often, it is necessary to
scale up the infrastructure for more computations and more data
transfers.
%
One approach is to buy more hardware, but it might be an expensive
solution.
%
As part of the \href{https://wiki.neic.no/wiki/Tryggve}{Tryggve
project}, we focused on an alternative approach, where we ask other
clusters, if they have available resources which we could
\emph{``borrow for a while''}.
%
We present here how we extended cloud resources across borders, and
whether there is a penalizing latency.
%
We stress-tested network and disk accesses. We further tested the
setup using two realistic workflows, with a surprising outcome:
computations hardly notice whether they are performed in one country
or another.
%
The latter aspect makes such a solution suitable for applications
requiring a secure environment, \eg programs that manipulate
sensitive data.
\end{abstract}
